\section{Summary of the Paper}

The paper proposes a general-purpose variational inference algorithm -- Stein Variational Gradient Descent (SVGD). In this section, we summarize the author's justification for the design of the SVGD algorithm. The pseudocode is in Algorithm \ref{alg:svgd}. The ultimate goal of the SVGD algorithm is to perform inference on a target distribution $p(x)$ on $\mathcal{X}=\mathbb{R}^d$, i.e. to estimate $\mathbb{E}_{x\sim p(x)}[f(x)]$ for some function $f(x)$, or to estimate the shape of the distribution $p(x)$. To do that, the algorithm constructs and updates multiple particles $x_1, \cdots, x_n$, which approximates a distribution $q(x)$ that has small KL divergence with regard to the target distribution $p(x)$. More specifically, they aim to solve the following problem.

% We first summarize the author's justification for the design of the SVGD algorithm, and then explain the extreme case, scalability, and performance of the SVGD algorithm.

% \subsection{Design of SVGD}

\begin{algorithm}[t!]
	\caption{Bayesian Inference via Variational Gradient Descent~\cite{ref_article_svgd}}
	\label{alg:svgd}
	\begin{algorithmic}
		\STATE {\bfseries Input:} A target distribution with density function $p(x)$ and a set of initial particles $\{x_i^0\}_{i=1}^n$.
		\STATE{\bfseries Output:} A set of particles $\{x_i\}_{i=1}^n$ that approximates the target distribution.
		\FOR {iteration $\ell$}
		\STATE{$x_i^{\ell+1}\leftarrow x_i^l + \epsilon_l \hat{\mathbf{\phi}}^*(x_i^\ell)$,
		
		where $\hat{\mathbf{\phi}}^*(x) = \frac{1}{n} \sum_{j=1}^n\left[ k(x_{j}^\ell, x)\nabla_{x_j^\ell}\log p(x_j^\ell) + \nabla_{x_j^\ell} k(x_j^\ell, x)\right]$}
		\ENDFOR
	\end{algorithmic}
\end{algorithm}

\begin{align}
    x_1,\cdots,x_n\sim q(x), \text{ where }q(x) = \arg\min_{q(x)} KL(q(x)\lVert p(x)).
\end{align}

The algorithm is designed with the following key ingredients.

\begin{enumerate}
    \item Iterative Particle Movement: under a small particle movement $T(x) = x + \epsilon \mathbf{\phi}(x)$, the transformed particles follow a push-forward distribution $q_{[\epsilon\mathbf{\phi}]}(x) = T_{\#}q(x)$ that captures the distribution of $T(x)$ for $x\sim q(x)$. Here $\epsilon$ is a small stepsize, and $\phi(x): \mathbb{R}^d\rightarrow \mathbb{R}^d$ could be any smooth mapping.
    \item Steepest descent direction and Stein's discrepancy: to find the mapping $\mathbf{\phi}(x)$ that results in the steepest descent of KL divergence, the author translate this objective to be the following Stein's discrepancy,
    \begin{align}
    \label{eqn:objective}
         \arg\max_\phi\frac{d}{d \epsilon}KL(q_{[\epsilon\mathbf{\phi}]}\left(x)\lVert p(x)\right) = \arg\max_\phi \mathbb{E}_{x\sim q}\left[trace\left(\mathcal{A}_p \mathbf{\phi}(x)\right)\right]
    \end{align}
    where the operator $\mathcal{A}_p \phi(x)= \nabla \log p(x) \cdot \phi(x) + \nabla\phi(x)$.
    \item Kernel trick for closed form solution: Let $\mathcal{H}$ be a reproducing kernel Hilbert space with positive definite kernel $k(x,x')$, i.e. $\mathcal{H}$ is the closure of the linear span $\{f: f = \sum_{i=1}^m a_i \cdot k(x, x_i), a_i\in \mathbb{R}, m\in \mathbb{N}, x_i\in \mathcal{X}=\mathbb{R}^d \}$. Then for $\phi(x) = \left(\phi_1(x), \phi_2(x), \cdots, \phi_d(x)\right)^T \in \mathcal{H}^d$, s.t. $\lVert\phi\rVert_{\mathcal{H}_d}\leq S(q, p)$, \eqref{eqn:objective}  has the following closed form solution.
    \begin{align}
    \label{eqn:solution}
        \phi(x) = \mathbb{E}_{x'\sim q}\left[ \mathcal{A}_p k(x', x) \right]
    \end{align}
    where $S(q, p) = \max_{\phi\in \mathcal{H}^d, s.t. \lVert\phi\rVert_2\leq 1} \mathbb{E}_{x\sim q}\left[trace\left(\mathcal{A}_p \mathbf{\phi}(x)\right)\right]$ is the Kernelized Stein's discrepancy. Here the authors assume that the function $K(x, x')$ given fixed $x'$ lies in the stein class of target density function $p(x)$.
    \item Approximation for $\phi$ with discrete particles $x_1, \cdots, x_n$: the closed-form solution \eqref{eqn:solution} involves high dimensional integration that is often intractable. Therefore, the author uses $n$ discrete particles $x_1, \cdots, x_n$ drawn from $q$ to perform the Monte Carlo estimate for the integral. In each iteration, the particle is then updated with the estimated mapping $T = I + \epsilon \phi$.
\end{enumerate}

The summation term in the iterative update step of the algorithm can be seen to have two parts. The first term is an attractive term that tries to maximize $\log p(x)$. The second term is a repulsive term that maximizes the difference between the two points through the kernel function. Balancing the two terms allow SVGD to pick samples with high $p(x)$ but are also different from each other.

% \subsection{Extreme case of SVGD with $n=1$ particle.} 

% The SVGD algorithm under one particle is equivalent to MAP estimate under gradient descent of $\log p(x)$ where $p(x)$ is the target distribution. \footnote{This is covered in the paper authors' blog \url{https://www.cs.utexas.edu/~qlearning/project.html?p=svgd}.} For example, under the RBF kernel $k(x, x') = \frac{1}{h}e^{\lVert x- x'\rVert_2^2}$, when applying the mapping $\hat{\phi}(x)$ in Algorithm \ref{alg:svgd} to the single particle $x=x_1^\ell$, we obtain the following gradient mapping.

% \begin{align}
%     \hat{\mathbf{\phi}}^*(x)|_{x=x_1^\ell} & = k(x_1^\ell, x)|_{x=x_1^e\ll} \nabla_{x_1^\ell}\log p(x_{1}^\ell) + \nabla_{x_1^\ell}k(x_1^\ell, x)|_{x=x_1^\ell}\\
%     & = k(x_1^\ell, x_1^\ell) \nabla_{x_1^\ell}\log p(x_{1}^\ell) + \mathbf{0}\\
%     & = \frac{1}{h} \nabla_{x_1^\ell}\log p(x_{1}^\ell)
% \end{align}

% \subsection{Performance of SVGD in terms of Estimation Error}


% \subsection{Scalability of SVGD with regard to model dimension and number of particles}